\documentclass[12pt]{article}

%% preamble: Keep it clean; only include those you need
\usepackage{amsmath}
\usepackage[margin = 1in]{geometry}
\usepackage{graphicx}
\usepackage{booktabs}
\usepackage{natbib}

%% for double spacing
\usepackage{setspace}

% for space filling
\usepackage{lipsum}
% highlighting hyper links
\usepackage[colorlinks=true, citecolor=blue]{hyperref}


%% meta data

\title{Analysis of MLB Reliever Leverage Effectiveness on Team Performance in the Statcast Era}
\author{William Quinlan\\
  Department of Statistics\\
  University of Connecticut
}

\begin{document}
\maketitle

\begin{abstract}
This is where I will put the abstract for my manuscript. I plan on doing a project about how the effectiveness of mlb relievers in leverage situations impacts team performance. 
\end{abstract}

\doublespacing

\section{Introduction}
\label{sec:intro}

Use this section to answer three questions:
Why is the topic important/interesting? These will get answered more in depth, but this topic is interesting because mlb teams are always looking to optimize performance and this is one facet of the game
What has been done on this topic in the literature? - There is a paper from 2016 on leverage situations, but the statcast era of baseball really only began around then so there is much more in depth data that can be used about pitcher metrics
What is your contribution? - I will bring in recent years of statcast data to see how leverage has impacted playoff performance since the last paper discussing this topic was written almost 10 years ago

\lipsum[1]

To cite a reference, here are examples.
\citet{xie2015dynamic} did something ... \lipsum[1]
Here is a source I am thinking of using
\citet{wdm2015consistency}

A lot of work has been done \citep[e.g.,][]{xie2015dynamic}.

\lipsum[2]

Some parametric bootstrap sample size approach was proposed by
\citep{dwivedi2017analysis}. 


% roadmap
The rest of the paper is organized as follows.
The data will be presented in Section~\ref{sec:data}.
The methods are described in Section~\ref{sec:meth}.
The results are reported in Section~\ref{sec:resu}.
A discussion concludes in Section~\ref{sec:disc}.


\section{Data}
\label{sec:data}

Use this section to describe the data that helps to answer your research
questions. Recall Einstein's equation
\begin{equation}
  \label{eq:mc2}
  E = m c^2,
\end{equation}
which states that the energy $E$ of a particle in its rest frame as the product
of mass ($m$) with the speed of light squared ($c^2$).
\lipsum{1}

\section{Methods}
\label{sec:meth}

Use this section to present the methodologies that will generate results by
analyzing the data. Suppose that the radius of a circle is $r$. Then its area is
\begin{equation}
  \label{eq:area}
  \pi r^2.
\end{equation}

Equation~\eqref{eq:area} is interesting. \lipsum[1]

Sometimes I don't want an equation to be numbered such as this one:
\[
  f(x) = \frac{1}{\sqrt{2\pi}} \exp\left( - \frac{x^2}{2} \right),
\]
which is the density of a standard normal variable.


\section{Results}
\label{sec:resu}

Table~\ref{tab:rv} summarizes some example draws from some distributions.
\lipsum[1]

\begin{table}[tbp]
  \caption{This is my first table.}
  \label{tab:rv}
\centering
\begin{tabular}{rrr}
  \toprule
normal & poisson & gamma \\ 
  \midrule
-0.110 & 4 & 2.401 \\ 
  0.116 & 4 & 3.529 \\ 
  -0.828 & 9 & 2.112 \\ 
  -0.066 & 6 & 11.104 \\ 
  0.219 & 3 & 4.815 \\ 
  0.303 & 5 & 2.188 \\ 
  0.544 & 0 & 8.050 \\ 
  -2.617 & 8 & 3.646 \\ 
  0.747 & 1 & 5.178 \\ 
  -1.103 & 4 & 3.043 \\ 
   \bottomrule
\end{tabular}
\end{table}

Figure~\ref{fig:histogram} shows the distribution of random normal data.


\begin{figure}[tbp]
  \centering
  \includegraphics[width=\textwidth]{histogram.pdf}
  \caption{This is my first figure.}
  \label{fig:histogram}
\end{figure}

\section{Discussion}
\label{sec:disc}

What are the main contributions again?

What are the limitations of this study?

What are worth pursuing further in the future?

\lipsum[1]
Watch for prevalence of diabetes \citep{wild2004global}.

\bibliography{refs}
\bibliographystyle{mcap}

\end{document}