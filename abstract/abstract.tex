\documentclass[12pt]{article}

%% preamble: Keep it clean; only include those you need
\usepackage{amsmath}
\usepackage[margin = 1in]{geometry}
\usepackage{graphicx}
\usepackage{booktabs}
\usepackage{natbib}

%% for double spacing
\usepackage{setspace}

% highlighting hyper links
\usepackage[colorlinks=true, citecolor=blue]{hyperref}


%% meta data

\title{Student Review of The Earth is Round $(p < .05)$}
\author{William Quinlan\\
  Department of Statistics\\
  University of Connecticut
}

\begin{document}
\maketitle

\begin{abstract}
Jacob Cohen's paper, \textit{The Earth is Round $(p < .05)$} is a critique of widespread misuses of hull hypothesis significance testing. Cohen argues that the emphasis placed on the arbitrary p-value threshold of 0.05 has hindered scientific progress. Cohen addresses questions of deductive logic, conclusions that can be drawn from information, an emphasis on practical significance, and much more. Cohen discusses the fact that p-values do not indicate whether the null hypothesis is true or false, and perhaps that in reality, the commonly used zero-effects null can never be truly false because there will always be some effects. Cohen places an emphasis on confidence intervals and effect sizes and states that these two measures ought to be presented alongside p-values. Cohen also discusses fallacious beliefs rejections will replicate. Cohen then discusses what we as researchers ought to do, focusing more on being a decective rather than a sanctifier. 

\end{abstract}

\doublespacing


\bibliography{refs}
\bibliographystyle{mcap}

\end{document}